%%%%%%%%%%%%%%%%%%%%%%%%%%%%%%%%%%%%%%%%%%%%%%%%%%%%%%%%%%%%%%%%%%%%%%%%%%%%%%%%%%%%%%%%%%%%%%%%%%%%%%%

\documentclass[rmp,aps,nofootinbib,superscriptaddress,floatfix]{revtex4-2} 
\usepackage{graphicx}
\usepackage{xcolor} 
\usepackage{amsmath}
\usepackage{amssymb} 
\usepackage{natmove}
\usepackage{natbib}
\usepackage{hyperref} 
\usepackage{bm}
\bibliographystyle{apsrev4-2}
\setcitestyle{numbers,square}

%%%%%%%%%%%%%%%%%%%%%%%%%%%%%%%%%%%%%%%%%%%%%%%%%%%%%%%%%%%%%%%%%%%%%%%%%%%%%%%%%%%%%%%%%%%%%%%%%%%%%%%

\begin{document}

\title{On the Origin of Shrimpoluminescence}

\author{Tyler C. Sterling}
\email{ty.sterling@colorado.edu}
%\affiliation{Department of Physics, University of Colorado at Boulder, Boulder, Colorado 80309, USA}

\date{\today}

\begin{abstract}
shrimpity shrimp shrimp shrimp. abstract abstract abstract abstract abstract abstract abstract abstract abstract abstract abstract abstract abstract abstract abstract abstract abstract abstract abstract abstract abstract abstract abstract abstract abstract abstract abstract abstract abstract abstract abstract abstract abstract abstract abstract abstract abstract abstract abstract abstract abstract abstract abstract abstract abstract abstract abstract abstract abstract abstract abstract abstract abstract abstract abstract abstract abstract abstract abstract abstract abstract abstract abstract abstract 
\end{abstract}

\maketitle

%\listoffigures
%\listoftables
%\tableofcontents

%%%%%%%%%%%%%%%%%%%%%%%%%%%%%%%%%%%%%%%%%%%%%%%%%%%%%%%%%%%%%%%%%%%%%%%%%%%%%%%%%%%%%%%%%%%%%%%%%%%%%%%

%\emph{Frequently Used Abbreviations}
%\begin{itemize}
%    \item SL: sonoluminescence
%    \item SBSL: single-bubble sonoluminesence
%    \item MBSL: multi-bubble sonoluminescence
%\end{itemize}


\section{Introduction}
Snapping shrimp, like our cute friend in Fig. \ref{fig:shrimp_claw} (left), have long been known to produce cavitating bubbles by \emph{snapping} their claws \cite{versluis2000snapping,lohse2001snapping,tang2019bioinspired}. They have a strong appendage (the \emph{dactyl}) in thier claw [Fig. \ref{fig:shrimp_claw} (center)] that is used to create a very high-velocity jet of water. The low pressure region in the jet's wake forms a bubble [Fig. \ref{fig:shrimp_claw} (t=0 ms)] that, among other things, produces an exceptionally loud noise when it collapses. The sound (i.e. the pressure wave) produced by even a \emph{single} shrimp's snap is detectable over a mile away \cite{everest1948acoustical}. The noise produced by groups of shrimp is so intense that the U.S. Navy used them as ``sonar-camoflauge" in the Pacific ocean during World War II \cite{versluis2000snapping}. 

\begin{figure}
\includegraphics[width=0.9\linewidth]{figs/shrimp_claw.pdf}
    \caption{(left) Snapping shrimp (\emph{Alpheus heterochaelis}). (center) Blown-up view of the shrimp's claw. The \emph{plunger} (pl) on the \emph{dactyl} (d) rapidly enters the \emph{socket} (s), ejecting a high-velocity jet of water. The water ejection and subsequent bubble formation and, finally, bubble collape (at t=0) are shown on the right. The units are ms. Adapted from ref. \cite{versluis2000snapping}}
\label{fig:shrimp_claw}
\end{figure}

The shrimp were not patriots helping the war-effort, however; they snapped for food. The shock-wave produced by the cavitating bubble is used to stun and even kill prey \cite{versluis2000snapping}. If the shrimp's prey had very sensitive eyes (and also were not dead) they might notice a flash of light is also produced through an effect referred to as ``shrimpoluminescence" in the case of the pistol shrimp \cite{lohse2001snapping}, but more generally known as \emph{sonoluminescence}.

Sonoluminescence (SL) is more precisely defined as the process by which a ``driven gas bubble collapses so strongly that the energy focusing at collapse leads to light emission" \cite{brenner2002single}. Sonoluminescence comes in two forms: (i) single-bubble sonoluminescence and (ii) multi-bubble sonolumiscence. The disticntion is self-explanatory: multi-bubble sonolumiscence (MBSL) consists of  ``the simultaneous creation and destruction of many seperate, individual cavitation bubbles" \cite{crum1994sonoluminescence,brenner2002single}, whereas in single-bubble sonoluminescence (SBSL), rather obviously, only a single bubble is present \cite{gaitan1992sonoluminescence}. The discovery of MBSL predates SBSL by $\sim$60 years but due to the more-or-less random and fleeting nature of the bubbles, SL in general wasn't well studied until the early 1990's when it was discovered that single bubbles could be created and periodically driven to produce light very high precisions \cite{crum1994sonoluminescence,gaitan1990experimental,gaitan1992sonoluminescence,brenner2002single}. We will discuss MBSL breifly later, but for now it suffices to say that nearly all theoretical and experimental progress has been made using SBSL, with some authors even calling it ``the hydrogen atom of sonoluminescence" \cite{lohse2018bubble,crum1994sonoluminescence}.

{\color{red}According to Suslick et al \cite{suslick2008inside}, SBSL makes it difficult to directly measure temperature and pressure in the bubble due to its tiny volume. On the otherhand, emission from a single bubble is not complicated from multiple scattering (off other bubbles), volume emission from the bubble cloud, etc. Similiarly, the bubble is not perturbed by interaction with other bubbles and, due to its tiny size, by interaction with the container walls.}

The discovery of SBSL created a rush of effort to explain the phenomena with arguments ranging from the simplest models \footnote{There are far too many to try list here and we will not be interested in the vast majority of early models which are now irrelevant. References to these can be found in Gaitan \cite{gaitan1990experimental} and Brenner et al. \cite{brenner2002single}.} to more exotic explanations based of quantum field theory \cite{schwinger1993casimir,eberlein1996sonoluminescence,liberati2000sonoluminescence}.


But besides the obvious case of shrimpoluminescence, why is SL interesting? Well, at a glance, it is not obvious why SL occurs. The acoustic wave in the liquid displaces molecules on the order of nanometers, costing an elastic energy of  $\sim1\times 10^{-12}$ eV/molecule while emitting visible light (through e.g. electronic transitions) costs an energy $\sim1$ eV/molecule \cite{lohse2018bubble,crum1994sonoluminescence}... there is a 12-orders-of-magnitude concentration of energy. That is huge! Estimates of the temperature at the center bubble are ...


%After a moments thought, however, the massive energy focusing is not that surprising. For example, the energy in the shock wave produced by a cavitating bubble, regardless of the light production, has been known to damage ship propellers since atleast 1917 when Lord Rayleigh's gave the first rigorous treatment of caviating bubbles \cite{gaitan1990experimental}. 

%So again, why is the light emmission interesting? In short, after 3 decades of studying SBSL, the mechanism of light production is still not understood. 

%The excitment wasn't restricted to academics however, with SBSL catching Hollywood's attention leading to an unpopular movie called \emph{Chain Reaction} where t


After 3 decades of studying SBSL, the physical mechanism of light production in SL is still not understood. This can be seen by glancing at the literature and noticing that there are many recent papers with different arguments for \emph{where} \footnote{We emphasive ``where" since it's not even certain if the light is \emph{surface} or \emph{volumetric} emission, i.e. it is literally not known \emph{where} the light comes from \cite{}.} the light comes from \cite{borisenok2020mechanisms,flannigan2013non,flannigan2012temperature,tatartchenko2017sonoluminescence}. Numerous aspects of the process are conveniently accessible to the experimentalist, while at the same time the theory of the bubble's interior is quite mature. Many theories require experimental inputs as parameters, while in other cases experimental results are indirectly \emph{inferred} by fitting to a theory. Put simply, the history of studying SBSL is a rather beautiful example of the scientific process of explaining nature. Accorind to Brenner ``SBSL has become a rather sophisticated testing ground for the ability of mathematical models and numerical simulations to explain detailed experimental data from a complicated physical process" \cite{brenner2002single}.


The goal of this paper is to catch the reader up on recent efforts to explain SBSL. While the author of this paper is particularly excited about shrimpoluminescence, it is important to stress that this work will focus on SBSL in general. It is apparently simpler to create and characterize SBSL in the laboratory without involving shrimp (for example, convincing the shrimp to snap requires tickling them \cite{lohse2001snapping,versluis2000snapping,lohse2018bubble}), so practically all work to study SL has not involved shrimp \footnote{ With notable exceptions \cite{tang2019bioinspired}}. Also notably absent will be any detailed discussion of MBSL since this will take us too far afield. Instead, we will summarize the history of SBSL research up to now starting at the discovery of MBSL leading to the subsequent discovery of SBSL and a rush of effort to explain it. We will explore both the experimental and theoretical discoveries along the way. Finally, we will review the current state of agreement between what is known experimentally and theoretically about SBSL. 

\section{Historical Overview}

SL was discovered by accident in 1933 (in the form of MBSL) by Marinesco and Trillat \cite{marinesco1933actions} and was subsequently characterized by Frenzel and Schultes \cite{frenzel1934luminescenz} \footnote{Neither the paper due to Marinesco et al. nor the one due to Frenzel et al. can be found by the author; in any case he can't read French or German so having the papers wouldn't be much use. As such, the story of how the effect was discovered is taken from more recent sources whose authors hopefully could read French and German \cite{brenner2002single,gaitan1990experimental,crum1994sonoluminescence}} Marinesco et al. were trying to accelerate photo development by \emph{insonating} developing fluid. They discovered that a photosensitive plate immersed in the insonated fluid became ``foggy" which they attributed to exposure to light. Shortly after, Frenzel et al. repeated the experiment and confirmed that the insonated fluid emits light in the form of a faintly glowing cloud of bubbles. 

This result was not particularly surprising to the community since it had been known for a while that cavitating bubbles could do tremendous damage to e.g. ships propellers. The discovery's impact is consicesly summarized by Brenner: ``if the cloud [of cavitating bubbles] collapses violently enough to break molecular bonds in a solid, why should it \emph{not} emit photons" \cite{brenner2002single}. In fact, cavitating bubbles had been of interest to engineers working on fluid mechanics for a while. The discovery of cavitation is credited to Euler (as early as 1754) who hypothesized that if the velocity in a fluid was large enough, negative pressures coud become so large as to ``break the fluid" \cite{li2015introduction,gaitan1992sonoluminescence} and was confirmed to exist (and named ``cavitation") in 1895 by engineers studying the failure of a British Navy ship's propeller \cite{li2015introduction}. Shortly after, Lord Rayliegh wrote down and solved the differential equation for a vapor filled cavity collapsing in water (the so-called Rayleigh equation), giving the first rigoruous theoretical treatment of cavitation \cite{rayleigh1917pressure,plesset1977bubble}. Rayleigh found that, for a bubble at a lower pressure than the surrounding fluid (and both pressures held constant), the bubble wall diverges during collapse. Still, very little information was accessible about the light emission until the 1990's when stable, single bubbles could be created and driven to emit light \cite{gaitan1990experimental,gaitan1992sonoluminescence,crum1994sonoluminescence}. With this discovery, serious interest took root. 

Details of a typical SBSL experiment will be given later, but for now it suffices to note that SBSL, unlike previous studies on MBSL, allowed very precise control and measurment of the SL process. With unprecidented experimental control, many discrepencies in previous assumptions about SL were discovered. Experiments measuring the duration of the light pulse found that it was orders or magnitude smaller than the time in which the bubble was compressed to its smallest radius \cite{barber1992resolving,barber1991observation}. This discovery implied that SL was nearly decoupled from the bubble's dynamics, contradicting models based on Rayleigh's equation. New models were proposed based on \emph{converging shock waves} at the bubble's center with estimates of the temperature at the center of bubbles $\sim10^8$ K \cite{wu1993shock,greenspan1993sonoluminescence}. At the same time, experimentalists fit the bubble's light emission spectra as a \emph{black body emitter} and concluded that the temperature in the bubble was atleast 25,000 K \cite{hiller1992spectrum}.

These very high estimates for the temperature in the bubbles lead to a spark of interest more broadly: if the temerature in the bubble were really that large, then it should be possible for nuclear \emph{fusion} to occur \cite{}. \textcolor{red}{ talk about the stupid ass movie and that asshole that faked science data.} 
Now talk about sonochemistry etc.




\section{What do we know?}
\subsection{The Bubble Wall}
As it turns out, the dynamics of the bubble in SBSL (regardless of the light emission) are quite well described both qualitatively and \emph{quantitatively} by the classical theory of bubble dynamics. This isn't particularly surprising since the light emission and bubble collapse occur at different timescales \cite{}. The starting point for any theory of the bubble dynamics is Rayleigh's original expounded upon by many others \cite{}: the so called ``the Rayleigh-Plesset" (RP) equation \footnote{The RP equation is derived from the Navier-Stokes equation in the appendix}:


\subsection{Experimental Results}
what conditions are necessary for light emission (i.e. what is the phase space \cite{lohse2018bubble}):
\begin{itemize}
    \item Ar content
    \item driving frequency
    \item P$_{max}$
    \item ...
\end{itemize}

\section{What don't we know}
\subsection{The Bubble Wall (Again)}
\subsection{Experimental Anomalies}


\section{Let there be light!}
\subsection{Thermal Arguments}
\subsection{Electric Arguments}

\subsection{The Casimir Effect}
\subsection{The Casimir Argument}


\section{Summary}

\section{Acknowledgments}

\appendix

\section{The Rayleigh-Plesset Equation}
The Rayleigh-Plesset (RP) equation can be deduced from the compressible Navier-Stokes equations \cite{prosperetti1999old,brenner2002single,prosperetti1986bubble,plesset1977bubble,suslick2008inside,yasui2018acoustic}. To that end, we first remind the reader of the Navier-Stokes equations. 

The \emph{total} change (in time) of mass in an arbitrary element of a fluid (whose coordinates and boundary may depend on time) is given by  $\frac{d}{dt} \int_{\Omega_t} \rho dV$ where $\Omega_t$ is the domain of the fluid element under consdiration and $\rho(\bm{x},t)$ is the local, time-dependent density of the fluid. Using the transport theorem of fluid dynamics \cite{mcdonough2009lectures}, Gauss's theorem, and (with only some loss of generality) taking $\Omega_t$ to be a \emph{fluid parcel} (so that the boundary of the fluid parcel, $\partial \Omega_t$, moves with the fluid's velocity, $\bm{u}(\bm{x},t)$), this becomes
\begin{equation}
    \frac{d}{dt} \int_{\Omega_t} \rho dV = \int_{\Omega_t} \partial_t \rho dV+\int_{\partial \Omega_t} \rho \bm{u} \cdot \bm{n} dA = \int_{\Omega_t} \partial_t \rho+\nabla \cdot(\rho \bm{u})dV.
    \label{eq:mass_continuity}
\end{equation}
Now to have \emph{mass-continuity}, which we will require on physical grounds, we set eq. \ref{eq:mass_continuity} to 0. Basically, we are saying no additional material is being added or removed from the fluid. Since the fluid parcel's boundary $\Omega_t$ is arbitrary, we find that $\partial_t \rho+\nabla\cdot(\rho \bm{u})\equiv 0$ which is true for every point in the fluid volume.

Newton's second law for fluids (\emph{momentum-continuity}) is defined similarly \cite{schoeffel2014lecture,mcdonough2009lectures}. We write the momentum-density as $\rho \bm{u}$ with momentum given by $\int_{\Omega_t} \rho \bm{u} dV$. To simiplify what follows, we focus on only the $i^{th}$ component of velocity at a time. Then Newton's second law can be written as 
\begin{equation}
    \frac{d}{dt} \int_{\Omega_t} \rho u_i dV =  \int_{\Omega_t} \partial_t (\rho u_i) dV+\int_{\partial \Omega_t} \rho u_i ( \bm{u} \cdot \bm{n}) dA = F_i.
\end{equation}
Now using Gauss's theorem, the product rule, and mass-continuity we find $ \frac{d}{dt} \int_{\Omega_t} \rho u_i dV =  \int_{\Omega_t} \rho \frac{d u_i}{dt} dV = F_i $. In vector notation this reads
\begin{equation}
    \int_{\Omega_t} \rho \frac{d \bm{u}}{dt} = \bm{F}
    \label{eq:mom_continuity}
\end{equation}
 
We now turn to the forces. Let us seperate the forces into \emph{body-} and \emph{surface-forces}: $\bm{F}=\int_{\Omega_t} \bm{f}_B dV+ \int_{\partial \Omega_t} \bm{f}_S dA $. The body-force density, $\bm{f}_B$, is due to forces acting in the bulk of the liquid e.g. gravity, electrostatic forces, etc. and is typically from sources external to the liquid. We won't be concerned with these forces here. Instead, let us concentrate on the suface part. If we only consider an ideal fluid, the surface part can be written as $\int_{\partial \Omega_t} \bm{f}_S dA =-\int_{\partial \Omega_t} p \bm{n} dA = -\int_{\Omega_t}\nabla p dV$ with $p$ the pressure  and the minus sign being due to the sign convention: \emph{out} of the surface $\partial \Omega_t$ is positive. 

If instead the fluid is real, we can't neglect viscous forces (i.e. dissipation). We suppose that the viscous surface term can be represented by a tensor, $\hat{\bm{\tau}}$ such that $\bm{F}_v=\int_{\partial \Omega_t} \hat{\bm{\tau}}\cdot \bm{n} dA = \int_{\Omega_t} \nabla \cdot \hat{\bm{\tau}} dV$. (Note, here the dot product is to be understood as matrix multiplication). To leading order in gradients of the velocity, (and under the physical constraints that the tensor be symmetric), the viscous-stress tensor is $\hat{\bm{\tau}}=\hat{\bm{\mu}}\cdot \hat{\bm{\epsilon}}$ where $\hat{\bm{\mu}}$ is the viscosity-tensor and $\hat{\bm{\epsilon}}$ is the strain-rate tensor: $\epsilon_{ij}=\frac{1}{2}\left[ \partial_i u_j+ \partial_j u_i \right]$. 

In general, $\hat{\bm{\mu}}$ is a symmetric rank-4 tensor. However, if we assume an isotropic fluid \footnote{Definitely true for water which is usually the fluid used in sonoluminescence experiments.}, we find that the the viscous-stress tensor can be seperated into two irreducible parts: a scalar, $\hat{\bm{\epsilon}}^{(v)}$, and traceless symmetric part, $\hat{\bm{\epsilon}}^{(s)}$ \cite{zee2016group,landau1987fluid}:
\begin{equation}
\begin{split}
    \hat{\bm{\tau}} & = \xi \hat{\bm{\epsilon}}^{(v)}+\mu \hat{\bm{\epsilon}}^{(s)} \\
    \hat{\bm{\epsilon}}^{(v)} & = \delta_{ij}\epsilon_{kk} \\
    \hat{\bm{\epsilon}}^{(s)} & = \hat{\bm{\epsilon}}-\frac{1}{3}\delta_{ij}\epsilon_{kk}.
    \label{eq:viscous_stress_tensor}
\end{split}
\end{equation}
Thus, we see that the viscosity-tensor only has two free components: $\xi$, the normal viscosity and $\mu$ the shear viscosity. It is usual to combine the pressure (i.e. elastic stresses) and the viscous-stresses into the stress-tensor $\hat{\bm{\sigma}}=-p\hat{\bm{I}}+\xi \hat{\bm{\epsilon}}^{(v)}+\mu \hat{\bm{\epsilon}}^{(s)}$. Combined with this, Newton's second law becomes
\begin{equation}
    \int_{\Omega_t} \rho \frac{d \bm{u}}{dt}-\bm{F} = \int_{\Omega_t} \rho \frac{d \bm{u}}{dt} - \nabla \cdot \hat{\bm{\sigma}}-\bm{f}_B dV = 0
\end{equation}
Again, since we allow the volume of integration to be arbitrary, the integrand must vanish everywhere: $\rho \frac{d \bm{u}}{dt} - \nabla \cdot \hat{\bm{\sigma}}-\bm{f}_B=0$. This is the momentum-continuity equation for a fluid. Now we would rather have this explicitly in terms of the velocity $\bm{u}$. To that end, we may subsitute eq. \ref{eq:viscous_stress_tensor} into the momentum-continuity equation. For e.g. the $i^{th}$ component we may expand out the divergence part as 
\begin{equation}
\begin{split}
    \left(\nabla \cdot \hat{\bm{\sigma}}\right)_i = -\partial_i p + \left[\xi +\frac{2}{3}\mu \right] \partial_i(\nabla \cdot \bm{u})+\mu \nabla^2 u_i \equiv -\partial_i p + \eta \partial_i(\nabla \cdot \bm{u})+\mu \nabla^2 u_i
\end{split}
\end{equation}
where we have introduced the \emph{bulk-viscosity} $\eta=\xi+\frac{2}{3}\mu$. Collecting all of the components into vector form and, combined with the mass-continuity equation \ref{eq:mass_continuity}, we arrive at the \emph{Navier-Stokes} (NS) equations for a compressible fluid:
\begin{equation}
\begin{split}
    \rho \frac{d \bm{u}}{dt} = \rho \left[ \partial_t \bm{u}+\left(\bm{u}\cdot \nabla \right)\bm{u} \right] & =-\nabla p + \eta \nabla \left(\nabla \cdot \bm{u} \right)+\mu \nabla^2 \bm{u} + \bm{f}_B \\ 
     \partial_t \rho+\nabla\cdot(\rho \bm{u}) & = 0.
     \label{eq:ns_equations}
\end{split}
\end{equation}

Some comments are in order. The total-derivative of a moving fluid $\frac{d}{dt}\equiv D_t$ is sometimes called the \emph{material derivative}. In the case of velocity $D_t \bm{u}$ is the \emph{acceleration} of the fluid. It records not only explicit time dependence of the fluid's velocity $\partial_t \bm{u}$ but also how the fluid's velocity varies \emph{in space} as it moves past us: $\left( \bm{u}\cdot\nabla \right) \bm{u}$. The pressure term on the right hand side is to be understood as forces from the \emph{elastic} energy. On the other hand, the terms $\sim \bm{u}$ are viscous (damping) terms that tend to make the velocity field spatially uniform: the vanish when the (spatial) derivatives of the velocity vanish. The bulk-viscosity term $\sim \eta$ damps radial changes in the fluids velocity (e.g. dilation/contraction) while the shear-viscosity term $\sim\mu$ resists shearing. The body term $\bm{f}_B$ is from external sources: from here on, we will assume it is zero. 

Even with these simplifying assumptions, solving eqs. \ref{eq:ns_equations} is a formidable task \footnote{In fact, it's not even clear that the NS equations \emph{can} be solved even for incompressible fluids. Proving that smooth, sensible solutions exist earns \$1,000,000 \cite{fefferman2006existence}. Better hurry though... inflation was 6.8\% in Nov. 2021} To that end, several more drastic approximations have to be made that allow us to make progess. 
\begin{enumerate}
    \item We will assume irrotational flow (i.e. $\nabla\times\bm{u}\equiv 0$) and only radial motion in the liquid (i.e. $\bm{u}=u\bm{r}$). Note that, since the flow is assumed irrotation, we can represent the velocity as the gradient of a scalar function: $u\bm{r}=\partial_r \phi \bm{r}$. This amounts to assuming that the bubble is always spherical. This seems like a rather drastic approximation but has been validated experimentally in many cases \cite{}. This can be understood by accounting for surface-tension at the liquid-bubble interface \cite{}. 
    \item Next, we assume that the viscous terms are negligible in the bulk dyanimcs of the liquid. We will acount, to some extent, for viscosity later when looking at the bubble-liquid interface. Moreover, we are assuming that the flow is \emph{isentropic}, i.e. that it is reversible (no damping) and that no heat is exchanged between fluid parcels. We will further assume that the liquid is isothermal, i.e. its temperature is constant (in space). Then pressure $p$ is determined from an instantaneous equation-of-state $p=p(\rho,T)$. In the case of SBSL, this is valid since the bubble makes up a \emph{tiny} fraction of the total volume and, as we will later see, heat-transport across the liquid-bubble interfacing is usually neglected anyway (i.e. the bubble is compressed adiabatically in the thermodynamic sense \footnote{According to Wikipedia, \emph{adiabatic} means \emph{fast} in thermodynamics lingo. This is relevant to us since we are claiming the bubble wall moves so quickly that it is compressed to its maximum pressure before any heat can flow out. On the otherhand, the mechanics lingo implies \emph{adiabatic} to mean \emph{slow}. This is, e.g. the adiabatic theorem in quantum mechanics: a perturbation acts so slowly that the system is in its groundstate at all times.}
    \item Related to the fact that the bubble is tiny, we assume that the extent of the liquid is so large compared to the bubble that we may consider the dynamics of the liquid as if there were no bubble present; similarly, we consider the dynamics of a bubble in an infinite, isotropic medium. Of course the bubble-liquid interface enters both systems as a boundary condition. 
\end{enumerate}







\newpage 

\section{random notes}

argon flash: a method of 'imaging explosions' similar to why argon is more useful for SBSL. argon has no internal DOF whereas air (i.e. N2, O2, water vapor) does and energy from explosions is absorbed dissociating bonds. argon, instead, becomes 'hotter' and emits more light, making photographing explosions easier. see wikipedia article "argon flash"

Brennen defines caviation as (bubble) nucleation when the pressure drops below the vapor pressure... as compared to boiling in which the temperature goes above the saturated liquid/vapor temp. \cite{brennen2014cavitation}. 

Yasui defines \emph{acoustic cavitation} as the formation and collapse of a bubble driven by a powerful ultrasonic wave \cite{yasui2018acoustic}. He defines bubble dynamics as bubble pulsation under the driving sound. \textcolor{red}{read this book}



some dickheads 3d printed a shrimp claw and it also shrimpoluminesces \cite{tang2019bioinspired}

apparently a shitty movie called 'Chain Reaction' with Keanu Reeves was centered around the premise of MBSL as a means of clean energy production. it was based on the speculation that temperatures inside the bubbles ciould reach up to millions of Kelvin and lead to fusion. this was debunked by Prosperetti (see the pop sci article \cite{chainreaction}). He points out that the temp is speculative and is based on assumptions of a perfect sphere collapsing. he notes that its probably not spherical when light is produced and even points out that groups have tried to produce fusion power by intentionally collapsing gas filled micro baloons with lasers... and failed. Tatartcheceko also shows that this is horsehit \cite{tatartchenko2017sonoluminescence}. in this article, Prosperetti also explains how a 'jet' of water whooshing across the bubble could be going faster that 4000 mph, fast enough to cause non-netwownian fracturing of the water leading to light producuction through (apparently) the same mechanism of light emession when ice or wint-o-green mints are snapped in half... huh. this is good introduction fodder. need to find the primary lit. 

for fusions, also see wikipedia article on 'bubble fusion' and 'Rusi P Taleyarkhan'. Ruse was some asshole prof at purdue that claimed fusion occured inside cavitating bubbles, but it was later found out that the faked data and interfered with peer reviews leading to him being fucking fired.

\emph{heurestic explanation of bubble trapping given by Brenner in \cite{chainreaction}.}: the bonds between molecules in a fluid are very strong, so instead of 'tearing', the fluid stretches. however, if there is a bubble present, the local dialation of the fluid can change the bubbles diameter by $\sim$ 1000 times to avoid breaking bonds at the interface between bubble and fluid. But when the compression waves comes by, the 'gas' in the bubble, which has boiled due to low pressure, offers very little resistance to compression. the wave compresses the bubble rapidly, which spikes the temp in the gas to incandescent temperatueres. this heuristic explanation is on-track with Crum \cite{crum1994sonoluminescence} below.

\section{1st discovery}
MBSL was discovered first. its (obviously) hard to create and control single bubbles so this wasnt feasible until the 1990s, attributed to Gaitan (see below). allegedly, it was Marinesco et al \cite{marinesco1933actions} that discovered MBSL in 1933 (according to \cite{gaitan1990experimental} below and to \cite{tatartchenko2017sonoluminescence}) but I cant find this paper (or understand it since its in french). Other sources (e.g.. wikipedia, \cite{tatartchenko2017sonoluminescence,crum1994sonoluminescence}) credit Frenzel et al \cite{frenzel1934luminescenz} as the discoverers. I cant German or find this paper either, but in anycase, Tatartchenko summarized thier discovery: Freznel and coworkers were trying to speed up photo development by 'insonating' i.e. subjected developer-fluid to sound. they found that when they immersed photo-plates or whatever in the fluid, it became 'foggy' and overexposed, as if subjected to diffuse light. A bunch of other papers followed up on this. get the deets from \cite{gaitan1990experimental} below. 

Update: Brenner \cite{brenner2002single} consisely summarizes the history: Marinesco notes the photographic plates fogs in an ultrasonic bath \cite{marinesco1933actions} and Frenzel et al \cite{frenzel1934luminescenz} deliberately studied the light emission. According to brenner, no one was surprised by this since it was already known that cavitating bubbles fuck shit up.

SBSL was discovered by Gaitan \cite{gaitan1990experimental,gaitan1992sonoluminescence} etc. who trapped a single air bubble in a water filled flask using piezoacoutics stuck to the outside. (note: the trapping force is called \emph{Bjerknes force}, refs in \cite{lohse2018bubble}).

\section{notes from \cite{crum1994sonoluminescence}}
pop sci type of magazine article. Gives heurestic expanation: When an intense sound field is produced within a liquid, microscopic cavities of gas or vapor can be generated when the liquid fails under tensile stress. The subsequent acoustic compression cycle forces these cavities to collapse violently, which results in a remarkable concentration of mechanical energy, estimated to be as high as 12orders of magnitude (refs are given too). Also notes that MBSL is very hard to study since the bubbles occur randomly and transiently. Describes SBSL as the \emph{hydrogen atom of sonoluminescence}.also discusses that attempts to measure the (time) interval over which light is produced have found it be less than $\sim$50 ps which is 1/1000 times smaller than theory predicts. They also found the light to be more 'stable' than the acoustic generator, i.e. more regular with more controlled intervals. an old but probably still relevant result. mentions a neat claim that the conditions of the gas at the center of the bubble are probably more like a metal than a gas. \textcolor{red}{hell, this article basically summarizes all of the anomalies present in 1994. a very good resource.}

\section{notes from ref \cite{lohse2018bubble}} 
SBSL was discovered by Giatan et. al. in 1990s at U. Mississippi \cite{gaitan1990experimental,crum1994sonoluminescence,gaitan1992sonoluminescence} (note, \cite{gaitan1992sonoluminescence} has $\sim$1000 citations..). The SBSL experiments were done by sticking peizoacoustic transducers to the outside of a water filled glass. the 'Bjerknes' force trapps the bubble and the acoustic wave drives its expansion/collapse. What Gaitan saw was that even the single bubble emits light. Wow! This was baffling since 'acoustic energy' \textcolor{red}{(i dont understand this)} is on the order of $\sim10^{-12}$ eV/molecule while visible light is $\sim1$ eV. NVM: I do get this. Typical acoustic waves, e.g. in air, only displace molecules on the order of nanometers, which aquires an elastic energy gaine between molecules of $\sim10^{-12}$ eV/molecule: certainly not enough to excite an electronic transition. 

This paper refs the review by Propsperetti: \cite{plesset1977bubble}. This paper gives EOM for the bubbles radius, the so called \emph{Plesset-Rayleight} equation which I have seen everywhere related to bubbles. Talks about solving it, some other crap related to bubble dynamics. this requires a detailed re-read.

Later in the paper, also revisists snapping shrimp.

\section{notes from ref \cite{prosperetti2004bubbles}}
has shit tons of relevant math. take a look. a pretty paper overall


\section{notes from \cite{gaitan1990experimental}}
disseration by Giatan who is credited with discovering SBSL. says apparently SL was discovered in 1933. need to go read this crap. Apparently this was by Marinesco et. al. in 1933. Zimakov tried to explain it in 1934 and concluded that the light was due to electric discharge between vapor cavities and the glass container. Frenzel and Schultes in 1935 instead claimed it was from friction between cavitating bubbles and the water. Chamber in 1936 studied SL in other nonwater liquids. Chamber also postulated the light was from 'tribolumiscence' which occurs when the crystalline structure of the liquid-bubble lattice collapses? Apparently somthing similar occurs in real crystals. ... shit, a bunch more models are given. give this dis. a lot of attention.

also says caviation was predicted by 'Euler' in 1754: \emph{if the velocity of a fluid is high enouhg, negative pressures can form and the liquid can 'break'.} the 'breaking' was called cavitation in 895 by R.E. Froude, whoever the fuck this is, when studying bubbles near propellers. Rayleigh, unsurpisingly, was the first to give a rigorous theoretical treatment of cavitation in 1917.

\section{notes from \cite{brenner2002single}}
defines SBSL as ``Single-bubble sonoluminescence occurs when an acoustically trapped and periodically driven gas bubble collapses so strongly that the energy focusing at collapse leads to light emission". 

\section{QED}
the ref \cite{liberati2000sonoluminescence} has lots of good refs. they have follow up papers extending Schwingers papers (which are refd therein) and referencing a bunch of other relevant literature. I should look for refs citing Schwingers paper to try to figure out the modern state of the art of this thin.

\section{Modelling}
in the ref \cite{schanz2012molecular}, they use MD to modelling the equation of state for a collapsing bubble. they calculate the temperature and study the conditions for emitting light.

\section{Refs. to different arguments}
ref \cite{didenko2000molecular} gives examples of other origin theories (molecular excited states, black-body radiation, bremsstrahlung, ion-electron recombination, confined electrons)


%%%%%%%%%%%%%%%%%%%%%%%%%%%%%%%%%%%%%%%%%%%%%%%%%%%%%%%%%%%%%%%%%%%%%%%%%%%%%%%%%%%%%%%%%%%%%%%%%%%%%%%

\bibliography{../ref}

\end{document}
