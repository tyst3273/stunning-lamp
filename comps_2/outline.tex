%%%%%%%%%%%%%%%%%%%%%%%%%%%%%%%%%%%%%%%%%%%%%%%%%%%%%%%%%%%%%%%%%%%%%%%%%%%%%%%%%%%%%%%%%%%%%%%%%%%%%%%

\documentclass[prb,aps,11pt,superscriptaddress,floatfix]{revtex4-2} 
\usepackage{graphicx}
\usepackage{color} 
\usepackage{amsmath}
\usepackage{amssymb} 
\usepackage{natmove}
\usepackage{natbib}
\usepackage{hyperref} 
\usepackage{bm}

%%%%%%%%%%%%%%%%%%%%%%%%%%%%%%%%%%%%%%%%%%%%%%%%%%%%%%%%%%%%%%%%%%%%%%%%%%%%%%%%%%%%%%%%%%%%%%%%%%%%%%%

\begin{document}

\title{The Y-files}

\author{Tyler C. Sterling}
\email{ty.sterling@colorado.edu}
\affiliation{Department of Physics, University of Colorado at Boulder, Boulder, Colorado 80309, USA}

\date{\today}

\begin{abstract}
This is my proposal for the project/paper portion of the \emph{Comps 2} (C2) exam in the physics graduate program at the University of Colorado Boulder (CU). I breifly summarize my history in the graduate program in order to inform the reader of my previous and current research activites; then I introduce my proposed topic. I provide historical context and a breif outline of the structure of my paper. The planned contents can be loosely group into the following 5 sections: (i) introduction and historical context, (ii) important theoretical details, (iii) experimental sucesses and failures, (iv) current state of the art, and (v) summary.  
\end{abstract}

\maketitle

%\listoffigures
%\listoftables
%\tableofcontents

%%%%%%%%%%%%%%%%%%%%%%%%%%%%%%%%%%%%%%%%%%%%%%%%%%%%%%%%%%%%%%%%%%%%%%%%%%%%%%%%%%%%%%%%%%%%%%%%%%%%%%%

\section{Previous Research History}
I have always been interested in doo-dads, gadgets, and gizmos. I grew up on a poor-ish farm in rural Texas. We couldn't afford to hire mechanics, plumbers, or electricitians, so we nearly everything ourselves. I was a curious enough kid, so I learned how to do these things too. 
    
My interest in mechanics, machninery, and DIY lead me to Texas State University to study Manufacturing. Skip forward a few years. I worked in a few labs as a undergraduate assistant: in my first gig, I programmed industrial robots to TIG weld test-specimens, ran the procedure, then measured the weld strengths. My lab had written some machine-learing type algorithms to automate the various parameters relevant to the weld strength (e.g. electrode travel speed, electrode distance from sample, current) and I was generating the training data (i.e. got two IEEE conference papers out of this work!). 

My next job was to manufacture some prototypes of heat sinks for Hyundai headlights. The eventual mass-produced things would be 3D printed, but we were going to machine them from fancy high-thermal conductivity alloys. I raised the question that the transport properties of these materials would be vastly different due to the different microstructures. I thought we should account for that and was given the go-ahead to start working on that problem. Unfortunately I graduated shorty after and left Texas State... but I didn't left my question behind. 

Skip forward another few years. I joined the materials science PhD program here at CU to study heat transport at the nano-scale. I wanted to understand how different crystalline structures in a material affect the flow of heat. I joined a project working on thermoelectric superlattices and worked with molecular dyanmics simulations to model and quantify the flow of heat in these structures. I eventually left this lab too (with one co-author publication in PRB), but not before I learned that the question I brought with me was probably not going to be easy to answer. Furthermore, I had learned that the lattice also exchanges heat with the electrons through electron-phonon coupling: I ended up with another question!

Finally, I landed in Dmitry Reznik's lab. My study in heat transport in crystals led me to focus a lot of my attention on phonons and I had become interetsted in electron-phonon coupling by now too. Dmitry and his lab are experts in this stuff and I was lucky enough to be invited to work with them. I wanted (and badly needed) to learn more advanced physics than my engineering background had provided, so Dmitry encouraged me to transfer into the physics PhD program. I was lucky enough to be let in... and here I am. In my time there, I have used density functional theory, molecular dynamics, and inelastic neutron scattering to study even wackier materials than I had ever guessed had existed. My personal and academic interests have evolved and by now I am heavily invested in applying (continously more advanced) computational methods to interesting materials science problems. I have calculated neutron scattering spectra from density functional theory to study electron-phonon effects in cuprates (two co-authro papers in PRB, first author paper in progress). In another project, I used time-of-flight inelastic neutron scattering data and data science-esque techniques and software developed by my lab to determine the phonon dispersions a spin-spiral ordered antiferromagnet (one co-authro Phys. Rev. Research paper). My current projects are involved in using molecular dynamics simulations to understand the lattice dynamics and electron-phonon effects in dynamically disordered CH$_3$NH$_3$PbI$_3$, which has a nearly-freely rotating molecules trapped in its lattice. We are also looking at the effects of static disorder on the lattice dynamics in a thermoelectric clathrate and the results are exciting. On the back-burner are experimental investigations (data already taken!) on nickelates and manganates. I have some ideas to use even more advanced theoretical/computation methods to include the non-trivial, dynamical magnetic ordering that couples with the lattice in these materials.

Now lets move on to my topic. Obviously, my interests and experiences lie in condensed matter so nearly anything in condensed matter is exlcuded as a topic choice. There are many fascinating and modern things here at CU that would probably be natural choices: e.g. quantum information, AMO physics, high-energy computational methods, etc. However, I suspect that a vast and many students jump onto these hot-topics and I have never been one to get-in-line. T that end, I have made it my goal to pick something that it is both fun and unusual. I have no personal interests in the \emph{paranormal}


\section{Paper Topic}

I am still tossing around 2 ideas: (i) physics of the paranormal and (ii) harvesting the zero-point energy.

\subsection{The Y-Files (or the physics of the paranormal)}
My proposed idea is to discuss the actual physics of the paranormal. Specifically, I would focus on only a few phenomena that really merit investigation. Things that I consider \emph{beyond realistic} (e.g. ghosts, monsters, and a whole slew of other things crammed into modern 'scary' movies) will not be considered. On the other hand, things that could convceivably fall under the realms of quantum mechanics, statistical mechanics, electrodynamics, thermodynamics, etc. (e.g. telekenesis, telepathy, etc.) will be looked at in some detail. The Y in the title is just the next letter in the alphabet after X: \emph{The X-files} was a TV show that followed an FBI Agent tasked with investigating reports of paramormal phenomena. Unlike the sometimes delusional, conspirary obsessed agent leading the show, we will not entertain explanations based on consipracies, aliens, or other unprovable or unverifiable non-sense.

Instead, we will undertake rigourous investigations grounded in physics. We will clearly define the phenomena under study. We will look at detailed experiments with controlled conditions as well as serious theories proposing plausibale explanations. A recurring theme will be to assess the thermodynamics of a phenomena: claims of occurences that violate our beloved laws of thermodynamics are probably false. On the otherhand, I wouldn't be so sure about phenomena like telekinesis and telepathy: we regularly experience and observe forces applied by 'invisible hands' in the form of electromagnetic interactions. Electric and magnetic fields can move massive amounts of energy and exert incredible forces. Yet, in many cases, we cannot detect the source of these phenomena unless we employ specialized instruments. Similarly, we send and recieve 'telepathic' messages in the form texts and emails on a, lets face it, nearly constant basis. 

To an ignorant observer these phenomena could seem unnatural, like science fiction. \emph{give a parable about our ancestors observing electricity, fire, etc.}. 


\section{layout}
Start with a discussion of how mythology and claims of the paranormal emerged as an explanation for things like electromagnetism, fire, etc. Maybe start with a neat parable. Move onto to highlight how rigourous physical emerged in the course of serious scientific investigation. Give examples of similar things in modern life and discuss the current state of the art theories about these things and point out the experimental situation and indicate the things that have been proved and disproven for-sure. Actually, maybe don't stick to 'paranormal' stuff but look at anything that is for certain real but unexplained. E.g. flashing/glowing lights in swamps? We will take a Sherlock Holme's in the one where goes to hunt the wolf in the bog approach. Maybe even like Fox Moulder. 



%%%%%%%%%%%%%%%%%%%%%%%%%%%%%%%%%%%%%%%%%%%%%%%%%%%%%%%%%%%%%%%%%%%%%%%%%%%%%%%%%%%%%%%%%%%%%%%%%%%%%%%

\bibliography{ref}

\end{document}
